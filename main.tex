% !TeX spellcheck = russian-aot
% !TeX encoding = UTF-8
% ----------------------------------------------------------------
% plc-sample-2017.tex
% Demonstration of PLC (Programming Languages and Compilers)
%   conference proceedings and articles template file
% http://plc.sfedu.ru/for-authors.html
% Main file
% Created: 20 January, 2017
% ----------------------------------------------------------------
%
% Requirements:
% - UTF-8 encoding;
% - Standard article document class;
% - A4 paper size;
% - 10pt font;
% - Single column;
% - Floating environments for figures (e. g.
%   \begin {figure} ... \end {figure});
% - Authors' emails (optionally) and affiliations in \author command;
% - An abstract and keywords are required;
% - If a BibTeX file is used, it must be in the form: <prefix>-biblio.bib
%   where <prefix> consists of the author's name and an abbreviated
%   organization name;
% - All used labels must be started with the same <prefix>;
% - Volume: - for proceedings: up to 2 pages; for articles: up to 8 pages.
%

\documentclass [a4paper] {article}

% ----------------------------------------------------------------
% Required packages

\usepackage [utf8] {inputenc}
\usepackage [T2A] {fontenc}
\usepackage [russian] {babel}

\usepackage {url}
\usepackage [style = gost-numeric] {biblatex}

% ----------------------------------------------------------------
% Title settings

\addbibresource {lukyanov-sfedu-biblio.bib}

% ----------------------------------------------------------------
% Title settings

\title %
  {Синтаксической анализ с контролируемыми побочными эффектами}

\author %
{%
  Лукьянов~Г.\,А.\textsuperscript {1}, \url {glukyanov@sfedu.ru} \and %
  Пеленицын~А.\,М.\textsuperscript {1}, \url {apel@sfedu.ru} \\
  \textsuperscript {1}Южный Федеральный Университет
}%

\date {}    % if desired

% ----------------------------------------------------------------
\begin {document}
% ----------------------------------------------------------------

\maketitle

\begin {abstract}
  %
  Целью работы является развитие гибких и выразительных методов построения функциональных синтаксических анализаторов на основе современных подходов к контролю вычислительных эффектов. В работе рассматривается язык 
  программирования Frank~\cite{Frank} и его применение к построению методов
  синтаксического анализа.  
  %
  \\ \textbf {Ключевые слова:} синтаксический анализ, парсер, монадические комбинаторы парсеров, функциональное программирование, вычислительные эффекты.
\end {abstract}

\section{Вычислительные эффекты}

Возможность статического контроля побочных эффектов является одним из
принципиальных преимуществ статически типизированных языков
программирования с богатыми системами типов. Отделение ``чистых''
(англ. \emph{pure}) функций от вычислений, обременённых
взаимодействием с устройствами ввода-вывода и другими побочными
эффектами, позволяет с большей уверенностью рассуждать о надёжности
разрабатываемой программы.

\section{Синтаксические анализаторы}

Синтаксический анализ может быть рассмотрен как вычисление с состоянием:
на старте парсеру подаётся некоторая входная последовательность символов --- 
начальное состояние. Каждый комбинатор каким-то образом модифицирует
состояние. Разбор завершается, когда входной поток символов полностью
исчерпывается. В случае обнаружения синтаксической ошибки или
неоднозначности, анализатор завершает работу и сообщает о проблеме.
Вышеописанные вычислительные эффекты могут быть выражены следующими
конструкциями языка~\texttt{Frank}.    

\begin{verbatim}
parse : {[Exception ParseError, State (List Char)] X
        } -> (List Char) -> Either ParseError X
parse p str = catch (state str p!)
\end{verbatim}

Простейшим парсером, на основе которого строятся все комбинаторы, является
парсер, разбирающий единственный символ входного потока. 

\begin{verbatim}
item : [Exception ParseError, State (List Char)]Char
item! = on get! { nil       -> throw inputExhausted
                | (x :: xs) -> put xs; x} 
\end{verbatim}

Парсер получает текущее состояние входного потока и, если он пуст, прерывает
работу с соответствующей ошибкой. Иначе --- обновляет состояние и возвращает
отщеплённый символ.

На основе парсера~\texttt{item} строится парсер, распознающий символ, который
соответствует заданному предикату.

\begin{verbatim}
sat : {Char -> [Exception ParseError, State (List Char)]Bool} -> 
      [Exception ParseError, State (List Char)]Char
sat p = on item! {c -> if (p c) {c} {throw parseError}}
\end{verbatim}

Другим важным комбинатором является альтернативное применение двух парсеров:
производится попытка применить первый парсер и, в случае неудачи --- второй.
Если же оба парсера не применимы --- процесс разбора завершается с ошибкой.

\begin{verbatim}
por : {[Exception ParseError, State (List Char)] X} -> 
      {[Exception ParseError, State (List Char)] X} -> 
      [Exception ParseError, State (List Char)] X
por p1 p2 = 
  on (parse p1 get!) { (right _)  -> p1! 
                     | (left  _)  ->  on (parse p2 get!) 
                       { (right _)   -> p2!
                       | (left err)  -> throw err
                       }}

\end{verbatim}

Для многократного повторения применения некоторого парсера необходимы
комбинаторы~\texttt{some} и~\texttt{many}. Первый применяет парсер-параметр
ноль или более раз, а второй --- хотя бы единожды. Типовые аннотации этих
парсеров идентичны.

\begin{verbatim}
some : {[Exception ParseError, State (List Char)]X} ->
       [Exception ParseError, State (List Char)](List X)
some p = p! :: many p

many p = por {some p} {nil}
\end{verbatim}

В качестве примера парсера верхнего уровня рассмотрим парсер для числовых
литералов, использующем в своей реализации все описанные выше комбинаторы.

\begin{verbatim}
number : [Exception ParseError, State (List Char)] (List Char)
number! = some {sat isDigit}
\end{verbatim}

\section{Выводы}

\printbibliography

% ----------------------------------------------------------------
\end {document}
% ----------------------------------------------------------------

\endinput

% End of File