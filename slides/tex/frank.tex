\begin{frame}{Язык программирования Frank}
  Последняя версия представлена на POPL'17 в статье Do be do be do от Sam Lindley, Conor McBride, Craig McLaughlin
  \begin{block}{Особенности}
    \begin{itemize}
      \item \textbf{Строгая} стратегия вычислений
      \item Алгебраические типы данных
      \item Разделение \emph{типов-значений} и \emph{типов-вычислений}
      \item \emph{Интерфейсы} --- сигнатуры эффектов
      \item \emph{Операции} --- обработчики эффектов
      \item Обычные функции являются тривиальными операторами, обрабатывающими пустой множество эффектов
      \item Полиморфизм эффектов основанный на \emph{охватывающего поля эффектов}
    \end{itemize}
  \end{block}
\end{frame}

\begin{frame}[fragile]{Типы данных и функции}
\begin{block}{Натуральные числа}
\begin{minted}{haskell}
data Nat = zero | suc Nat
\end{minted}
\end{block}
\pause
\begin{block}{Рекурсивные функции}
\begin{minted}{haskell}
eqNat : Nat -> Nat -> Bool
eqNat zero zero  = true
eqNat (suc x) (suc y) = eqNat x y
eqNat zero _     = false
eqNat _  zero    = false
\end{minted}
\end{block}
\end{frame}

\begin{frame}[fragile]{Полиморфные по эффектам функции}
\begin{block}{Функция \texttt{map}}
\begin{minted}{haskell}
map : {X -> Y} -> List X -> List Y
map f nil         = nil
map f (cons x xs) = cons (f x) (map f xs)
\end{minted}
\end{block}
\pause
\begin{block}{Применение \texttt{map} в отсутствие эффектов}
\begin{minted}{haskell}
pureMap : {List Int}
pureMap = map {x -> x + 1} [1,2,3]
\end{minted}
\end{block}
\end{frame}


\begin{frame}[fragile]{Полиморфные по эффектам функции}
\begin{block}{Эффект консольного ввода-вывода}
\begin{minted}{haskell}
interface Console = inch : Char
                  | ouch : Char -> Unit
\end{minted}
\end{block}
\pause
\begin{block}{\texttt{map} в поле эффекта \texttt{Console}}
\begin{minted}{haskell}
effectfulMap : {[Console]List Unit}
effectfulMap = map ouch "Hello World!"
\end{minted}
\end{block}
\pause
\begin{block}{Тип \texttt{map} без синтаксического сахара}
\begin{minted}{haskell}
map : {{X -> []Y} -> List X -> []List Y}
\end{minted}
\end{block}
\end{frame}



\begin{frame}[fragile]{Управление потоком выполнения}
\begin{block}{Энергичный условный оператор}
\begin{minted}{haskell}
iffy : Bool -> X -> X -> X
iffy tt t f = t
iffy ff t f = f
\end{minted}
\end{block}
\pause
\begin{block}{``Традиционный'' условный оператор}
\begin{minted}{haskell}
if : Bool -> {X} -> {X} -> X
if tt t f = t!
if ff t f = f!
\end{minted}
\end{block}
\pause
\begin{block}{Аналог оператора монадического связывания \texttt{>>=}}
\begin{minted}{haskell}
on : X -> {X -> Y} -> Y
on x f = f x
\end{minted}
\end{block}

\end{frame}

\begin{frame}[fragile]{Эффект изменяемого состояния}
\begin{block}{Интерфейс эффекта \texttt{State}}
\begin{minted}{haskell}
interface State S = get : S
                  | put : S -> Unit
\end{minted}
\end{block}
\pause
\begin{block}{Обработчик эффекта \texttt{State}}
\begin{minted}{haskell}
state : S -> <State S>X -> X
state _ x          = x
state s <get -> k>  = state s (k s)
state _ <put s -> k>  = state s (k unit)
\end{minted}
\end{block}
\end{frame}

\begin{frame}[fragile]{Исключения}
\begin{block}{Интерфейс эффекта \texttt{Exception}}
\begin{minted}{haskell}
interface Exception E
  = exception : E -> Zero
\end{minted}
\end{block}
\pause
\begin{block}{Возбуждение исключения}
\begin{minted}{haskell}
throw : E -> [Exception E]X
throw e = on (exception e) {}
\end{minted}
\end{block}
\pause
\begin{block}{Обработка исключения}
\begin{minted}{haskell}
catch : <Exception E>X -> Either E X
catch x = right x
catch <exception e -> _> = left e
\end{minted}
\end{block}

\end{frame}



